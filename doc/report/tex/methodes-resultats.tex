\chapter{Méthodes et résultats}\label{chapter:methodes-resultats}

Dans ce chapitre, nous présentons la méthode que nous avons utiliser pour obtenir des résultats au niveau de l'IA du jeu.

\section{Méthodes d'obtention des résultats}

Afin de réaliser notre projet d'IA41, nous avons procédé de la manière suivante:

\begin{itemize}
    \item Implémentation de l'algorithme du negamax sans élagage alpha beta.
    \item Écriture d'une heuristique sommaire de test.
    \item Tests approfondis de l'algorithme pour vérifier son fonctionnement en jouant contre l'IA\@.
    \item Écriture de nouvelles heuristiques.
    \item Faire s'affronter deux IAs en faisant varier l'heuristique et la profondeur de l'arbre min-max pour voir laquelle est
        la meilleure.
\end{itemize}

Après celà, nous avons implémenté un algorithme d'élagage apha beta, afin d'accélérer la « vitesse de reflexion » des différentes
IA\@. L'élagage alpha beta reste sans grand effet lors de l'utilisation de l'heuristique de base, car celle-ci est quasi binaire
et n'offre donc que peu de possibilité de coupures.

\section{Résultats}

Au fur et à mesure de l'implémentation des différents éléments de l'IA, nous avons obtenu divers résultats :

\begin{itemize}
    \item Nous avons découvert que le premier joueur a toujours moyen de gagner entre 4 et 6 coups
        (on a au début cru à une anomalie de l'algorithme nega-max).
    \item Il est aussi possible de tomber dans des cycles (ou états stables) où aucun joueur ne peut gagner.
    \item Nous avons pu classer les heuristiques selon le nombres de parties gagnées contre telle ou telle heuristiques à
        telle ou telle profondeur en faisant un mini tournoi d'IA\@. C'est là qu'on leur a donné leur nom « facile », « normal », …
    \item Il est aussi clair que l'IA est meilleure que le joueur à partir d'une certaine profondeur en particulier pour les
        heuristiques « difficile » et « légendaire ».
\end{itemize}

