\chapter{Méthodes et résultats}\label{chapter:methodes-resultats}

\section{Méthodes d'obtention des résultats}

    Afin de réaliser notre projet d'IA41, nous avons procédé de la manière suivante:
    
    \begin{itemize}
        \item Implémentation de l'algorithme du negamax sans élagage alpha beta.
        \item Écriture d'une heuristique sommaire de test.
        \item Tests approfondis de l'algorithme pour vérifier sonb fonctionnement en jouant contre l'IA\@.
        \item Écriture de nouvelles heuristiques.
        \item Faire s'affronter deux IAs en faisant varier l'heuristique et la profondeur de l'arbre min-max pour voir laquelle est
            la meilleure.
    \end{itemize}

    Après celà, nous avons implémenté un algorithme d'élagage apha beta, afin d'accélérer la ``vitesse de reflexion'' des différentes
    IA\@. L'élagage alpha beta reste sans grand effet lors de l'utilisation de l'heuristique de base, car celle-ci est quasi binaire
    et n'offre donc que peu de possibilité de coupe.

\section{Résultats}

    \(\rightarrow\) le premier joueur a toujours moyen de gagner entre 4 et 6 coups
    (on a au début cru à une anomalie de l'algorithme nega-max).

    \(\rightarrow\) classement des heuristiques.

    \(\rightarrow\) l'IA est meilleure que le joueur.

