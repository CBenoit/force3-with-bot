\chapter{Structures de données utilisées}\label{chapter:structures-donnees}

Dans ce chapitre, nous expliquons les différentes structures de données utilisées pour représenter le jeu.

\section{Couleur d'un jeton}

La couleur d'un jeton (ou d'un joueur) est simplement une constante de type \emph{square::type} (donnée
sous forme d'une enum en C++).

Le code se trouve dans le fichier \emph{src/logic/boardsquare.hpp}.

\section{Plateau de jeu}\label{struct:boardstate}

La structure représentant un plateau de jeu a été optimisé pour prendre le moins de
place possible en mémoire. En effet, lors de l'exploration de l'arbre par l'ia,
on peut devoir créer des centaines, voire des miliers de plateaux de jeux différents.

Le plateau de jeu s'est donc finalement vu réduire à un \emph{uint\_fast32\_t}, c'est à dire
32 bits sur une architecture 32 bits, et 64 sur une architecture 64 bits. \\

En effet, chaque case ne peut être que dans quatre états différents:
\begin{itemize}
    \item Bleu
    \item Rouge
    \item Blanc (disponible)
    \item Noir  (trou)
\end{itemize}

Il faut donc 2bits minimum pour représenter chaque case. Comme on a 9 cases différentes,
le minimum d'espace nécessaire pour stocker un plateau est de $2 × 9 = 18$bits.

Finalement, le plateau est représenté par les 18bits de poids faible de l'entier, et on ignore
les bits de poids forts. On a de plus $Blanc = 00$, $Noir = 01$, $Rouge = 10$ et $Bleu = 11$:

\begin{figure}[H]
    \centering
    \begin{tabular}{|c|c|c|}
        \hline
        Bleu  & Blanc & Rouge \\
        \hline
        Bleu  & Noir  & Blanc \\
        \hline
        Rouge & Bleu  & Rouge \\
        \hline
    \end{tabular}
    \enskip
    \begin{tabular}{|c|c|c|}
        \hline
        11 & 00 & 10 \\
        \hline
        11 & 01 & 00 \\
        \hline
        10 & 11 & 10 \\
        \hline
    \end{tabular}
    \enskip
    \begin{tabular}{|c|}
        \hline
        00000000 \ldots 0000 11 00 10 11 01 00 10 11 10 \\
        \hline
    \end{tabular}
    \newline
    \caption{Différentes façons de représenter un même plateau de jeu.}
\end{figure}

\section{Coup à jouer}

Un coup à jouer est représenté par une structure.
Il y a trois catégories de coup possibles :
\begin{itemize}
    \item \emph{déplacement par glissement}, dans le code: \emph{Slide}.
    \item \emph{déplacement d'un pion de sa couleur}, dans le code: \emph{Swap}.
    \item \emph{pose d'un pion}, dans le code: \emph{SetColor}.
\end{itemize}

Chacun de ces mouvements a sa propre structure de données ce qui permet de les distinguer facilement.
\begin{itemize}
    \item \emph{Slide}, position de départ et position d'arrivée.
    \item \emph{Swap}, position de départ et position d'arrivée.
    \item \emph{SetColor}, position à laquelle poser le pion.
\end{itemize}

Le code se trouve dans le fichier \emph{src/logic/move.hpp}.

Pour simplifier certaines opérations, nous avons fait une classe \emph{MoveWrapper} qui permet de contenir
n'importe quel type de coup. Le code associé se trouve dans le fichier \emph{src/logic/movewrapper.hpp}.

\section{État du jeu}

Un état du jeu est représenté par :
\begin{itemize}
    \item le nombre de pions restant pouvant être placés par chaque joueur.
    \item la couleur du joueur dont c'est le tour.
    \item un état de plateau (cf \ref{struct:boardstate}).
    \item le dernier mouvement effectué pour atteindre l'état courant.
\end{itemize}

Le code se trouve dans le fichier \emph{src/logic/gamestate.hpp}.

\section{Arbre générique}

Pour représenter les états successifs du Force 3, nous avons voulu utiliser un
arbre générique, c'est à dire un arbre pouvant avoir un nombre non défini
d'enfant par noeud.

Nous avons choisi d'utiliser un arbre car il n'est pas important de savoir si
l'on est retourné dans l'état précédent.

\begin{center}
    \begin{forest}for tree={inner sep=0pt,outer sep=0pt}
    [1
      [2
        [9]
        [10]
        [11]
        [12]
      ]
      [3
        [13]
        [14]
        [15]
        [16]
      ]
      [4]
      [5
        [17]
        [18]
        [19]
        [20]
      ]
      [6]
      [7
        [21]
        [22]
        [23]
        [24]
      ]
      [8
        [25]
        [26]
        [27]
        [28]
     ]
    ]
    \end{forest}
\end{center}

Cet arbre est construit sur le principe des itérateurs, permettant d'insérer
facilement des éléments dans l'arbre.
Il dispose de plusieurs types d'itérateurs:
\begin{itemize}
    \item Un permettant de traverser l'arbre en profondeur, en commencent avec
    le noeud le plus à gauche.
    \item Un permettant de traverser l'arbre en profondeur sur une profondeur
    donnée, en commencent par le noeud le plus à gauche.
    \item Un permettant de visiter tout les enfants d'un node.
\end{itemize}

La gestion de la mémoire est effectué par l'allocateur standard du C++,
permettant d'initialiser la mémoire et d'appeler les constructeurs séparément.

Cependant, cet arbre ne s'est pas révélé utile, car le stockage des différents
états n'est pas nécessaire et a un impact mémoire très élevé.

Le code se trouve dans le fichier \emph{src/common/tree.hpp}.

