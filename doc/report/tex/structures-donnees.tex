\chapter{Structures de données utilisées} \label{chapter:structures-donnees}

Structures de données pour représenter le jeu.

\section{Plateau de jeu}

…

\section{Couleur d'un jeton}

…

\section{État du jeu}

…

\section{Coup à jouer}

…

\section{Arbre générique}

Pour représenter les états successifs du Force3, nous avons voulu utiliser un
arbre générique, c'est à dire un arbre pouvant avoir un nombre non défini
d'enfant par noeud.

\begin{center}
\begin{forest}for tree={inner sep=0pt,outer sep=0pt}
[1
  [2
    [9]
    [10]
    [11]
    [12]
  ]
  [3
    [13]
    [14]
    [15]
    [16]
  ]
  [4]
  [5
    [17]
    [18]
    [19]
    [20]
  ]
  [6]
  [7
    [21]
    [22]
    [23]
    [24]
  ]
  [8
    [25]
    [26]
    [27]
    [28]
 ]
]
\end{forest}

Cet arbre est construit sur le principe des itérateurs, permettant d'insérer
facilement des éléments dans l'arbre.
Il dispose de plusieurs types d'itérateurs:
\begin{itemize}
    \item Un permettant de traverser l'arbre en profondeur, en commencent avec
    le noeud le plus à gauche.
    \item Un permettant de traverser l'arbre en profondeur sur une profondeur
    donnée, en commencent par le noeud le plus à gauche.
    \item Un permettant de visiter tout les enfants d'un node.
\end{itemize}

La gestion de la mémoire est effectué par l'allocateur standard du c++,
permettant d'initialiser la mémoire et d'appeler les constructeurs séparément.

Cependant, cet arbre ne s'est pas révélé utile, car le stockage des différents
états n'est pas nécessaire et a un impact mémoire très élevé.
\end{center}
