\chapter{Annexes}

Le code complet des deux algorithmes suivants se trouvent dans le fichier \emph{src/logic/ai.cpp} du projet.

\section{Code de prise de décision}\label{src:think}

\lstinputlisting[language=C++]{./src_samples/think.cpp}

La méthode \emph{think} est appelé pour demander à l'IA de prendre une décision en choisissant le prochain coup à
jouer.

Le type \emph{heuristic::return\_t} représente un score d'heuristique.

\section{Code du nega-max avec élagage alpha-béta}\label{src:negamax}

\lstinputlisting[language=C++]{./src_samples/negamax.cpp}

Le type \emph{heuristic::return\_t} représente un score d'heuristique.

\section{Code de l'heuristique difficile}\label{src:heuristic_hard}

\lstinputlisting[language=C++]{./src_samples/heuristic_hard.cpp}

La fonction prend en paramètre le joueur et trois pions et retourne le score associé.
Le code complet se trouve dans le fichier \emph{src/logic/heuristic.cpp} du projet.

Le type \emph{return\_t} représente un score d'heuristique.

