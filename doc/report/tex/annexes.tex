\chapter{Annexes}

Le code complet des deux algorithmes suivants se trouvent dans le fichier \emph{src/logic/ai.cpp} du projet.

\section{Code de prise de décision}\label{src:think}

\lstinputlisting[language=C++]{./src_samples/think.cpp}

La méthode \emph{think} est appelé pour demander à l'IA de prendre une décision en choisissant le prochain coup à
jouer.

Le type \emph{heuristic::return\_t} représente un score d'heuristique.

\section{Code du nega-max avec élagage alpha-béta}\label{src:negamax}

\lstinputlisting[language=C++]{./src_samples/negamax.cpp}

Le type \emph{heuristic::return\_t} représente un score d'heuristique.

\newpage
\section{Code de l'heuristique « normal »}\label{src:heuristic_normal}

\lstinputlisting[language=C++]{./src_samples/heuristic_normal.cpp}

La fonction \emph{is_there_a_connected_token} prend en paramètre l'état du plateau, la couleur du joueur actuel,
et la position du pion pour lequel on veut vérifier si un autre pion de la même couleur est présent sur une case adjacente.
La fonction retourne true le cas échéant.
Le code complet se trouve dans le fichier \emph{src/logic/heuristic.cpp} du projet.

Le type \emph{return\_t} représente un score d'heuristique.

