\chapter{Analyse du problème}

Dans ce chapitre, nous allons détailler la phase d'étude qui a précédé le
développement de notre solution.

Dans ce sujet, nous devions réaliser une intelligence artificielle pour
jouer au Force 3.
Le Force 3 peut être représenté par une succession d'états. Cependant il est
possible de retourner dans un état précédent selon les actions des joueurs.
Nous devons donc utiliser un arbre au lieu d'un graphe.
De plus même si la complexité combinatoire du jeu n'est pas très élevée, on ne peut quand même pas lister tous les états
en un temps raisonnable.

\section{Problématique}

Compte tenu de ce qui a été dit précédemment, nous nous sommes penchés sur la problématique suivante: \\
« Comment réaliser une intelligence artificielle jouant au force 3 avec un temps de réflexion raisonnable » \\
Afin de répondre à cette problématique nous avons analysé les solutions suivantes.

\section{Minimax}

Dans un premier temps, nous avons étudié le minimax afin de répondre à notre
problématique.
Cependant, nous avons préféré utiliser une simplification du minimax le négamax.

\section{Négamax}

Le négamax, étant juste une simplification du minimax, requiert les même conditions,
le jeu considéré doit être un « jeu à somme nulle », c'est à dire un jeu où si
l'action est positive pour un joueur alors elle doit être négative pour l'autre
joueur, ce qui est le cas du Force 3. De plus, le jeu doit être à informations
complètes, ce qui signifie que dans notre cas, toute les informations doivent
être contenus dans le plateau de jeu.

Comme toute les conditions sont réunies, nous pouvons utiliser le négamax (valide aussi pour le minimax).

\section{Négamax avec élagage alpha-bêta}

Malgré le fait que le négamax soit adapté au Force 3, il ne peut répondre seul à
la problématique, en effet, nous souhaitons que le joueur puisse se frotter à une IA très forte
s'il en a envie, c'est à dire que l'IA doit pouvoir travailler avec un arbre très profond. Nous envisagons donc
de rajouter l'élagage alpha-bêta pour respecter la contrainte de temps même avec un arbre profond.

