\chapter{Étude du problème}

Dans ce chapitre, nous allons détailler la phase d'étude qui a précédé le
développement de notre solution.

Dans ce sujet, nous avons devions réaliser une intelligence artificielle pour
jouer au force 3.
Le Force 3 peut-être soit représenté par une succession d'état. Cependant on
peut retourner dans un état précédant, de plus même si la complexité du jeu
n'est pas très élevé, on ne peut quand même pas lister tout les états.

\section{Problématique}

Compte tenu de ce qui a été dit précédemment, nous nous sommes penchés sur la
problématique suivante: \\
« Comment réaliser une intelligence artificielle jouant au force 3 en temps
réel ? » \\
Afin de répondre à cette problématique nous avons analysé les solutions
suivantes.

\section{Minimax}

Dans un premier temps, nous avons étudié le minimax afin de répondre à notre
problématique.
Cependant, nous avons préféré utiliser une simplification du minimax le négamax.

\section{Négamax}

Le négamax étant une simplification du minimax, requis les même conditions,
le jeu considéré doit être un «jeu à somme nulle», c'est à dire un jeu où si
l'action est positive pour un joueur alors elle doit être négative pour l'autre
joueur, ce qui est le cas du force 3. De plus, le jeu doit être à information
complète, ce qui signifie que dans notre cas, toute les informations doivent
être contenus dans le plateau de jeu.

Comme toute les conditions sont réunis, nous avons choisi d'utiliser le négamax.

\section{Négamax avec élagage alpha-bêta}

Malgré le fait que le négamax soit adapté au force 3, il ne peut répondre seul à
la problématique, en effet, nous avons du rajouter l'élagage alpha-bêta pour
respecter la contrainte temps réel. En effet si l'on augmente la profondeur dans
le négamax, l'intelligence artificielle met trop de temps à réfléchir ce qui ne
permet pas au joueur humain d'apprécier pleinement la partie.

En rajoutant l'élagage alpha-bêta, on peut ainsi augmenter la profondeur et
améliorer la capacité d'analyse de l'intelligence artificielle tout en gardant
un temps de réflection similaire à celui d'un humain.

%  LocalWords: alpha-bêta minimax négamax
